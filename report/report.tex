\documentclass[10pt,twocolumn]{article}

% Följande rad ska göra det möjligt att använda svenska bokstäver, som å, ä, ö. Kravet är 
% då att filen sparas i UTF-8-format. Om detta inte fungerar för dig, så kan du alltid 
% använda dig av {\aa} för å, \"a för ä och \"o för ö.
\usepackage[utf8]{inputenc}

% Följande väljer typsnitt som är kloner av Times New Roman, Helvetica och lämpliga till
% dem anpassade matematiktypsnitt.
\usepackage{newtxtext}
\usepackage{newtxmath}

%  Följande tillhandahåller miljön spver­ba­tim som är lämplig för att typsätta programkod.
\usepackage{spverbatim}

\raggedbottom
\sloppy

\title{Laborationsrapport i TSKS10 \emph{Signaler, Information och Kommunikation}}

\author{Malcolm Vigren \\ malvi108, 19950127-0970 }

\date{xxx maj, 2017}

\begin{document}

\maketitle

\section{Inledning}

Syftet med denna laboration var att IQ-demodulera och behandla en radiosignal
för att kunna höra vad som sändes. Enligt problemformuleringen skickar
radiostationen signalen:
\begin{align*}
    x(t) = x_I(t)\cos(2 \pi f_c t) - x_Q(t)\sin(2 \pi f_c t) + w(t) + z(t)
\end{align*}
där $x_Q(t)$ och $x_I(t)$ är de meddelanden som ska lyssnas efter. De
innehåller båda två olika melodier samt var sitt ordspråk som ska identifieras.
$f_c$ är signalens bärfrekvens, $z(t)$ är en summa av andra I/Q-modulerade
signaler och $w(t) = 0.001(\cos(2 \pi f_1 t) + \cos(2 \pi))$, där $f_1$ och
$f_2$ ligger långt ifrån bärfrekvenserna hos de andra signalerna. En del av
uppgiften är att ta reda på $f_c$, $f_1$ och $f_2$.

På grund av utformningen hos miljön innehåller den mottagna signalen en
ekoeffekt, sådan att vi tar emot signalen:
\begin{align*}
    y(t) = x(t - \tau_1) + 0.9x(t - \tau_2)
\end{align*}
En del av uppgiften är att ta reda på fördröjningen mellan de olika signalerna,
alltså $\tau_2 - \tau_1$.

Signalen är sparad i en \textit{wav}-fil med sampelfrekvensen 400kHz.

\section{Metod}

Uppgiften löstes på följande sätt...

\section{Resultat}

Den sökta informationen är:
\begin{itemize}
\item Bärfrekvensen för nyttosignalen är $f_c=94$ kHz.
\item $f_1=141500$ Hz och $f_2=141501$ Hz
\item $\tau_2 - \tau_1 = 420$ ms.
\item Ordspråket i $x_I(t)$ är "Väck inte den björn som sover".
\item Ordspråket i $x_Q(t)$ är "Äpplet faller inte långt från trädet".
\end{itemize}

\clearpage

\section*{Min Matlab-kod:}
\begin{spverbatim}
clear all
close all

for k=1:...
  ...
end

plot(...,...)
\end{spverbatim}

\end{document}
