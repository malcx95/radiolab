\documentclass[10pt,twocolumn]{article}

% Följande rad ska göra det möjligt att använda svenska bokstäver, som å, ä, ö. Kravet är 
% då att filen sparas i UTF-8-format. Om detta inte fungerar för dig, så kan du alltid 
% använda dig av {\aa} för å, \"a för ä och \"o för ö.
\usepackage[utf8]{inputenc}

% Följande väljer typsnitt som är kloner av Times New Roman, Helvetica och lämpliga till
% dem anpassade matematiktypsnitt.
\usepackage{newtxtext}
\usepackage{newtxmath}

%  Följande tillhandahåller miljön spver­ba­tim som är lämplig för att typsätta programkod.
\usepackage{spverbatim}

\raggedbottom
\sloppy

\title{Laborationsrapport i TSKS10 \emph{Signaler, Information och Kommunikation}}

\author{Malcolm Vigren \\ malvi108, 19950127-0970 }

\date{xxx maj, 2017}

\begin{document}

\maketitle

\section{Inledning}

Denna laboration gick ut på att...

\section{Metod}

Uppgiften löstes på följande sätt...

\section{Resultat}

Den sökta informationen är:
\begin{itemize}
\item Bärfrekvensen för nyttosignalen är $f_c=94$ kHz, $f_c=132$ eller $f_c=56$ kHz
\item $f_1=141500$ Hz och $f_2=141501$ Hz
\end{itemize}

\clearpage

\section*{Min Matlab-kod:}
\begin{spverbatim}
clear all
close all

for k=1:...
  ...
end

plot(...,...)
\end{spverbatim}

\end{document}
